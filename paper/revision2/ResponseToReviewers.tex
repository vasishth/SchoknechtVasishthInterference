\documentclass[12pt]{article}
\usepackage[a4paper,top=2cm,bottom=2cm,left=2.5cm,right=2.5cm,marginparwidth=1.75cm]{geometry}
%\geometry{landscape}                % Activate for for rotated page geometry
%\usepackage[parfill]{parskip}    % Activate to begin paragraphs with an empty line rather than an indent
\usepackage{graphicx}
\usepackage{float}
\usepackage{subfigure}
\usepackage{rotating}
\usepackage{booktabs}
\usepackage{amssymb}
\usepackage{amsmath}
\usepackage{amsthm}
\usepackage{color}
\usepackage{gb4e}
\noautomath

%\usepackage[backend=biber,useprefix=true,style=apa,url=false,sorting=nyt,eprint=false]{biblatex}

%\addbibresource{interferencebib.bib}

\usepackage[natbib=true,maxcitenames=2,bibstyle=authoryear, style=apa]{biblatex}
\usepackage{lineno,xcolor,clipboard,graphicx}


\usepackage{csquotes}
\DeclareLanguageMapping{american}{american-apa}
\bibliography{interferencebib}

\usepackage{marginnote}
\setcounter{section}{-1}

\openclipboard{output-reviews}

\renewcommand{\qedsymbol}{\rule{0.7em}{0.7em}}

\usepackage{epstopdf}
\newcommand{\revised}[1]{{\color{black}{#1}}}

\DeclareGraphicsRule{.tif}{png}{.png}{`convert #1 `dirname #1`/`basename #1 .tif`.png}

\title{Response to editor and reviewers (JML-24-17)}
\author{Pia Schoknecht and Shravan Vasishth}
%\date{}                                           % Activate to display a given date or no date


\usepackage{xr}

\makeatletter
\newcommand*{\addFileDependency}[1]{% argument=file name and extension
\typeout{(#1)}% latexmk will find this if $recorder=0
% however, in that case, it will ignore #1 if it is a .aux or 
% .pdf file etc and it exists! If it doesn't exist, it will appear 
% in the list of dependents regardless)
%
% Write the following if you want it to appear in \listfiles 
% --- although not really necessary and latexmk doesn't use this
%
\@addtofilelist{#1}
%
% latexmk will find this message if #1 doesn't exist (yet)
\IfFileExists{#1}{}{\typeout{No file #1.}}
}\makeatother

\newcommand*{\myexternaldocument}[1]{%
\externaldocument{#1}%
\addFileDependency{#1.tex}%
\addFileDependency{#1.aux}%
}

\myexternaldocument{elsarticle-template-harv}

\begin{document}
\hfill \today
%\section{}
%\subsection{}

\noindent \textbf{Prof.\ Jarrold}\\
\noindent \textbf{Associate Editor}\\
\noindent \textbf{Journal of Memory and Language}
\vskip 1em

\noindent \textbf{Response letter for Submission JML-24-17, 2nd round of revisions}
\vskip 2em
\noindent Dear Prof. Jarrold, 
\vskip 1em
Together with my co-authors Shravan Vasishth and Himanshu Yadav, I would like to submit our second revision of the article titled ``Do syntactic and semantic similarity lead to interference effects? Evidence from self-paced reading and event-related potentials using German''.
We are grateful for thoughtful and constructive comments and suggestions from yourself and the reviewers, and feel that the quality of the manuscript has been greatly improved by the revisions.

%add description of revisions here

Below, we separately address each comment from the reviewers and yourself. All changes have been highlighted in the revised manuscript. The actionable parts of the comments are highlighted in bold.

\vskip 1em
\noindent Sincerely,
\vskip 1em
\noindent Pia Schoknecht\\
Postdoctoral researcher\\
Department of Linguistics\\
University of Potsdam, Germany
\thispagestyle{empty}
\newpage

\section*{Editor's comments} 
\subsection*{Comment E.1}
\begin{quote}
``Dear Dr. Schoknecht,

Thank you for submitting your revised manuscript to the Journal of Memory and Language. This has now been reviewed by the same three experts who reviewed the previous submission, and you can find their comments below.

You will see from these that all three reviewers appreciate the thoroughness with which you responded to the issues raised on the initial submission. Reviewers 1 and 3 raise some remaining points but recommend acceptance, while Reviewer 2 is less persuaded that your revisions have adequately addressed their concerns. Given this set of reviews, I have come to a `revise' decision because I would like you to have one more go at addressing the remaining points made by all three reviewers.

I do not anticipate sending out the next version for review again, but rather anticipate being in a position to accept it at that point. However, it's important to note that that is not guaranteed, and one reason I have come to a `revise' rather than `accept subject to minor revisions' is that, like Reviewer 1, I was unable to access your data via your OSF link. \textbf{It is very important to us at JML that data are readily accessible, and that accessibility needs to be in place before any potential acceptance decision}. 
''
\end{quote}

\subsubsection*{Response to E.1}
\textcolor{red}{We have made the OSF repository public and updated the link in the manuscript.}


\subsection*{Comment E.2}
\begin{quote}
``The other key points I would like to ask you to pay particular attention to in any response to the current set of reviews are:

\textbf{Reviewer 2 and 3's clear view that you still overstate the implications of your findings in places}.''
\end{quote}

\subsubsection*{Response to E.2}
\textcolor{red}{add *in this design* whenever we say that there is no syn interference}

\subsection*{Comment E.3}
\begin{quote}
``\textbf{Reviewer 2's additional point 1 that asks whether you have changed that particular analysis}.''
\end{quote}

\subsubsection*{Response to E.3}
Yes, we changed all analyses in the 1st revision. As we had explained in the response to reviewers on August 29, 2024 (specifically in the response letter and our responses to the comments E.2 and R3.5), the priors were changed from directional priors in the original version to symmetrical priors in the revised version.


\subsection*{Comment E.4}
\begin{quote}
``and \textbf{their final point about the length of the current version of the manuscript}.''
\end{quote}

\subsubsection*{Response to E.4}
\textcolor{red}{remove repetitions}

\section*{Reviewer \#1} 

\subsection*{Comment R1.1}
\begin{quote}
``I was the Reviewer \#1 of the previous version of the manuscript.

All my concerns are addressed in this revision, so I am happy to say I don't see reasons to ask for resubmission or for another round of major, significant revisions. I found it a solid paper in the first round already, but with the revisions and especially given the modified and extended analysis, I find the whole story more convincing and also, fair.

I do have some suggestions. In my view these are only minor. They mainly concern just the way things are presented or discussed. I list my comments by page.

p. 16: osf data etc.

I wanted to check those, but \textbf{when I clicked the link, it said I don't have permissions and that I need to request access}.''
\end{quote}

\subsubsection*{Response to R1.1}
\textcolor{red}{We have made the OSF repository public and updated the link in the manuscript.} 


\subsection*{Comment R1.2}

\begin{quote}
``p. 17, SPR experiment

\textbf{Did the analysis include any cut-off point for reading times to exclude data? I could not find this in the manuscript. If it did, could this information be included?} If not, I would strongly suggest to use this. In my own experience with Prolific, I noticed the following very common pattern: people basically do not read around 10-20\% of experimental items - that is, for quite a significant subset of participants I noticed that they have super fast reading times on some subset of items, most likely because they just rushed through those items by just pressing the space bar and keeping it pressed. This is sometimes followed by a long waiting time at some other point, so those participants would not be outliers when we check the total time in the experiment. These people and their data would not be excluded by checking their answers to comprehension questions, especially if only a small subset of items has comprehension questions, like one third, as in this experiment (basically, they could still easily pass with around 90\% success rate). So if this was not checked, it would be good to check and \textbf{consider removing RTs based on small/large cut-off points and to see whether that affects the analysis. If this was done already, it would be good to report more details, in particular, the cut-off points and also the amount of data that was removed this way}.''
\end{quote}

\subsubsection*{Response to R1.2}
\textcolor{red}{We trimmed the reading times (between X and Y ms) before analysis and have now added this information to the manuscript.}

\textcolor{red}{Paste text}

\subsection*{Comment R1.3}
\begin{quote}
``p. 36: ``Regarding the possibility that sentence structure confounds caused the
effects in the pre-critical region(.)"

\textbf{I wonder whether it would make sense to briefly summarize what the sentence structure confound should be.} Otherwise the whole paragraph is hard to follow for readers who did not read or do not remember Mertzen et al. (2023).''
\end{quote}
\subsubsection*{Response to R1.3}
\textcolor{red}{We have added the following text to the manuscript on page X:}


\subsection*{Comment R1.4}
\begin{quote}
``p. 51: ``For more discussion of the role of structural retrieval cues, see Franck and Wagers (2020) and Arnett and Wagers (2017)."

\textbf{It would be good to cite Kush et al. (2015) here} (after all, a significant portion of Franck and Wagers, 2020, builds on Kush et al. 2015 to explore ways to deal with structural cues).

Kush et al. (2015) Relation-sensitive retrieval: Evidence from bound variable pronouns. Journal of memory and language.''
\end{quote}
\subsubsection*{Response to R1.4}
We have added the suggested reference (now on page \pageref{cite_kush}):

\begin{quote}
    \Paste{cite_kush}
\end{quote}



\subsection*{Comment R1.5}
\begin{quote}
``p. 55: ``$\Psi(d(x_i, y), 0 , \delta)$, where $\Psi(.|\delta)$"

I was a bit confused here, since the formula  ``$\Psi(d(x_i, y), 0 , \delta)$ " did not have any  ``$\Psi(.|\delta)$" part.''
\end{quote}

\subsubsection*{Response to R1.5}
\textcolor{red}{Opened issue and assigned to Himanshu}

\subsection*{Comment R1.6}

\begin{quote}
``p. 61, \textbf{Figure 14: one value is outside the scale of the bottom left graph. If possible, it would be good to make it visible}.''
\end{quote}

\subsubsection*{Response to R1.6}
\textcolor{red}{Check how the figure changes}

\subsection*{Comment R1.7}
\begin{quote}
``p. 64: ``Word-by-word presentation is likely to be more demanding on the comprehender's memory, which should make it easier to detect interference effects. So, the differences in methodology do not offer a straightforward
explanation for the lack of syntactic interference in our data compared to the
previous studies."

Actually, I thought here the authors basically provided an explanation of the differences between their findings and the previous findings because of differences in the methodology, even though their last sentence said otherwise. Let me spell out it: since the current methodology (SPR, EEG) is likely more demanding on memory, it is possible that people give up on detailed processing; rather, they do something akin to good-enough processing and do not fully process; consequently, they could still get interference from semantics (since the semantic interference is really simple, it is just lexical semantics and readers don't need to fully parse for that) but they don't correctly parse embedded subjects, hence the syntactic interference is gone. \textbf{The authors in fact come to good-enough processing in Section 7.3, but then it is not linked back to differences between their findings and the other findings (which I think should be, I do think this is a possible explanation).}''
\end{quote}

\subsubsection*{Response to R1.7}
\textcolor{red}{Follow the suggestion.}


\subsection*{Comment R1.8}
\begin{quote}
``p. 76: \textbf{Chromy should appear like this: ``Chromý" (note the correct diacritics above the "y")}

Thank you for your manuscript!''
\end{quote}
\subsubsection*{Response to R1.8}
We have corrected the spelling of Jan Chromý's last name.

\section*{Reviewer \#2} 
\subsection*{Comment R2.1}
\begin{quote}
``I appreciate the authors' responses to my previous comments. However, despite the revisions, significant weaknesses remain, rendering in my opinion this study primarily a methodological contribution that adds little value to our understanding of interference effects. Additionally, the authors' strong claims often come through as overstatements that could result in serious misconceptions.
My overall assessment is as follows.

1. Interpretability of the SPR results: The results from the self-paced reading (SPR) task remain unclear. The only observable effect—the semantic interference—emerges well before the critical region of interest (specifically, at the distractor region) and persists without change beyond the critical region. As a result, any effect at the critical region is uninterpretable.

The authors acknowledge the interpretability issue with the SPR results in their response to my comment, and they now include the following statement: ``Given the pre-critical reading time differences, effects in the later regions, i.e., the critical and spill-over regions, cannot be attributed clearly to the stimuli in these regions and sentence processing mechanisms associated with them. Therefore, we refrain from further discussing effects occurring in the later regions." \textbf{However, this comes after more than 10 full pages (pp. 25-37) of discussion on the reaction times (RTs) in the critical and pre-critical regions, which would lead any reader to believe these findings are meaningful. If, as the authors now realize, the effect emerges much earlier, this entire discussion is misleading: the SPR results after the distractor region are simply uninterpretable.}''
\end{quote}

\subsubsection*{Response to R2.1}
We agree with the review that the discussion of the SPR results was misleading and have revised it accordingly. We now start the discussion by discussing the effects at the distractor (see Response to R2.2) and then briefly discuss the later effects. 

\textcolor{red}{Revise text and paste here}

\subsection*{Comment R2.2}
\begin{quote}
``However, there are also problems with the interpretation of the effect at the distractor region. If the authors want to attribute this effect to encoding interference, then they must argue for the existence of long-lasting encoding effects, as this very same effect persists throughout the entire sentence. Yet, the mechanism behind such enduring encoding effects remains unclear, and the authors do not address this challenge. Feature overlap, a commonly proposed mechanism for encoding interference, seems unlikely. It would suggest that the two elements compete for the same feature throughout the entire sentence, which is not plausible. Activation leveling, another potential mechanism, is also improbable, as it equalizes the activation of competing elements and should result in similar reaction times at some point, which we do not see here. \textbf{What is evident from these findings is that the inanimate distractors are read faster than the animate ones (but the opposite pattern is found in ERPs, see point 3), and this preference creates a reaction time difference that persists until and beyond the critical verb. If the authors wish to interpret this as evidence of encoding interference, they must provide compelling reasons to support the idea that encoding interference can have such a prolonged effect, along a clear mechanism for it}.''
\end{quote}
\subsubsection*{Response to R2.2}
The focus of the present study was to investigate retrieval interference due to syntactic and semantic similarity. The finding that the reading times were likely affected by encoding interference and not by retrieval interference was unexpected (but see \citeauthor{mertzen}'s (\citeyear{mertzen}) discussion). Given that the present manuscript contains two large-sample experiments and computational modeling (and its substantial length), we think that an in-depth discussion of encoding interference is beyond its scope. To address the reviewer's concern that encoding interference would not cause such long-lasting effects as found in our SPR data, we have added the following text to the manuscript about a recent joint account of encoding and retrieval interference:

\textcolor{red}{Write someting about Himanshu's work in relation to this}
 
\subsection*{Comment R2.3}

\begin{quote}
``2. Lack of syntactic interference evidence: No evidence for syntactic interference was found in either task, contradicting two previous studies that tested similar structures (Van Dyke 2007; Mertzen et al. 2023). \textbf{In my previous review, I noted that the syntactic manipulation used was not appropriate and thus unlikely to induce interference effects. Although the authors concede that this might be true, they did not address the logical implications of this acknowledgment, as evidenced by their repeated strong assertions}, such as the one in the abstract: ``Surprisingly, in both experiments, Bayes factor analyses showed evidence against interference due to syntactic cues".
If the syntactic manipulation is irrelevant, then repeatedly concluding that syntactic interference is absent is not only an overstatement but also misleading. All this study allows us to conclude is that when cues that are irrelevant to syntax are manipulated (like [+being a subject of whichever clause at whichever level of embedding], underpowered studies (Van Dyke 2007; Mertzen et al. 2023) may falsely report interference effects. While this is a valid conclusion and a sound methodological critique, it does not significantly advance our theoretical understanding of syntactic interference. Specifically, it fails to demonstrate that when relevant syntactic cues are manipulated, syntactic interference is not attested. Yet, this is the conclusion the authors consistently imply throughout their manuscript, leading to significant misconceptions. As I mentioned in my previous review, beyond the two studies the authors are building on, numerous studies in the literature show that hierarchically intervening elements do generate strong syntactic interference.
It is a fine enterprise to correct inaccurate claims from previously underpowered studies that used irrelevant manipulations; however, this should not not suggest that the findings presented here have a broader significance than they do.''
\end{quote}
\subsubsection*{Response to R2.3}
We agree with the reviewer that our present work does not rule out syntactic interference in general, but that the present data just suggests that the specific design used in this work did not induce considerable syntactic interference. We have made the following edits in the manuscript to emphasize this point:

\textcolor{red}{revise text and paste here}

\textcolor{red}{SV email: i think we didn't write clearly enough about the fact that syntactic interference *in this design* is expected given the previous work. we did say it but needs to be clearer. also, when we say we find evidence against syn int, we should always add, in this design. and then we should say that this does not mean that there is no syntactic manipulation that would cause no interference, there could be. i think that point is valid. what i found strange in her comment was that she says both that van dyke and mertzen found syn int, but also that syn int is never going to show up in this design because it is not the right manipulation.  maybe her point is that we only show that the van dyke and mertzen result was Type M error (i.e., there is no real effect in this design).}
 
\subsection*{Comment R2.4}
\begin{quote}
``3. Misalignment of SPR and ERP results: The results from the self-paced reading (SPR) and event-related potential (ERP) methods do not align, with the only observed effect—semantic interference—pointing in opposite directions: inhibitory in SPR but facilitatory in ERP. Specifically, the authors report a reversed effect in the P600 component, showing a reduced P600 in conditions of higher semantic interference. This contradicts typical predictions, where the P600 is usually greater under high interference. The authors interpret this as a facilitatory interference effect (i.e., easier processing with high semantic interference). However, if this were the case, we would also expect faster reaction times (RTs) in high semantic interference conditions, while the opposite is observed in the SPR data.

4. Conflicting directions in ERP components: Within the ERP results, the different components also point in opposite directions. The P600 component shows a reversed effect, with a reduced P600 in conditions of high semantic interference. However, the N400 component follows the expected pattern, with stronger activation under high semantic interference. It remains unclear how the directionality of these two can be reconciled within the same theoretical framework.''
\end{quote}
\subsubsection*{Response to R2.4}
\textcolor{red}{Not sure whether this comment requires action.}

 
\subsection*{Comment R2.5}
\begin{quote}
``5. Unclear conflicting mechanisms: The authors attribute the SPR results to encoding interference and the ERP results to retrieval interference, leading to a contradictory conclusion. I won't delve further into this point, as the issues raised earlier already highlight more fundamental problems.''
\end{quote}
\subsubsection*{Response to R2.5}
\textcolor{red}{Not sure whether this comment requires action. Maybe 1-2 sentences about Himanshu's work.}
 
\subsection*{Comment R2.6}
\begin{quote}
``Additional points:

1. Original version: ``By contrast, the syntactic interference effect had only anecdotal evidence under the two narrow priors (Normal-(0, 0.1): BF10 = 2.5, Normal-(0, 0.5): BF10 = 1.2). There was evidence against syntactic interference under wider priors (Normal-(0, 1): BF10 = 0.66, Normal-(0, 5): BF10 = 0.14). What this means is that, in our data, only if we assume a priori that the effect size is relatively small, there is very weak evidence in favor of syntactic interference."

Revised version: ``By contrast, Bayes factors provided either no evidence for or even evidence against syntactic interference in both spatiotemporal windows (BF10 $<$ 1.2). Similarly, Bayes factors provided either no evidence for or even evidence against the interaction in both spatio-temporal windows (BF10 $<$ 1.2)."

\textbf{Did the authors change the analyses? Why what was reported as anecdotal evidence under narrow priors is now reported as no evidence or evidence against?}''
\end{quote}
\subsubsection*{Response to R2.6}
Yes, the priors were changed in the 1st revision (directional priors in the original version and symmetrical priors in the revised version, see also our Response to E.3). This resulted in small numerical changes of the results, e.g., BF10 = 2.5 in the original analysis decreased to BF10 = 1.2 in the revised analysis. Furthermore in the original version of the manuscript, we were inconsistent in our description of small BFs, i.e., BF10 = 1.2. Since the first revision and in line with \textcite{jeffreys1998theory} and \textcite{lee2014bayesian}, we now consistently interpret such small BFs as providing no evidence. 

 
\subsection*{Comment R2.7}
\begin{quote}
``2. As a side note, on p. 58, the authors conclude that ``the parser searches for a subject that is within the same clause but uses the animacy cue without reference to the clause in which a noun appears." Does this mechanism truly seem plausible to the authors? What would this imply in practice? \textbf{Does it suggest that semantic and syntactic information are entirely encapsulated from each other?} If we were to set aside the issues I raised earlier, this would be one of the paper's central conclusions, yet the authors fail to provide a credible mechanism to support it.''
\end{quote}
\subsubsection*{Response to R2.7}
\textcolor{red}{SV email: the syn / sem comment is also interesting because there are some theories that only assume semantic cue association--there isn't even any syntax. some discussion of the theoretically possible connections between syn and sem processing would help. e.g., one extreme is that syn and sem operate in tandem (e.g., categorial grammars), others believe that sem processing strictly follows syntactic structure building (classical montagovian stuff), then we have the sem only models (e.g., sentence gestalt model). there is no consensus on what the relationship is between syn and sem, so these could operate independently. local coherence is also an example--if we assume syntactic processing and no independent semantic processing, local coherence would not arise (depending on how one assumes that syntactic processing works).}
 
\subsection*{Comment R2.8}
\begin{quote}
``3. How do the fillers differ from the experimental items? In the example the authors reported there is retrieval at place again:

Experimental: The neighbor believed that the widower, who had told her that the loss was awful, regularly drank in the evenings to forget.\\
Filler: The carpet maker, who came to the workshop early, repaired the especially beautiful old carpet while listening to the news.

The authors state, ``The fillers were less syntactically complex than the experimental items but had at least one embedded clause and generally provided more variety." \textbf{In what way did they provide more variety? In which respects? Both fillers and test items involve retrieval; what is the role of the fillers in this context?}''
\end{quote}
\subsubsection*{Response to R2.8}
\textcolor{red}{Describe why fillers should be comparable in complexity to the experimental items and how they distracted the participants from the purpose of the experiment}
 
\subsection*{Comment R2.9}
\begin{quote}
``4. There are numerous repetitions throughout the paper. A prime example is the paragraph on ``hyp testing using BF" (p. 45), which is repeated almost verbatim in the ``Discussion" section (p. 47). This redundancy occurs several times, making the paper excessively and unnecessarily long. The authors should streamline the content to enhance clarity and conciseness.''
\end{quote}

\subsubsection*{Response to R2.9}
\textcolor{red}{remove repetitions}


\section*{Reviewer \#3} 

\subsection*{Comment R3.1}
\begin{quote}
``This is a revision of a paper that I previously reviewed. (I was Reviewer \#3 in the previous round.)

In the previous round of reviews, I was generally happy with the authors' findings. My queries were mostly related to the interpretation and generalizability of those findings. This is a routine issue in psycholinguistic research. A study looks at effects in one very specific (and often complicated) sentence type, and then draws conclusions about very broad notions such as `syntax' and `semantics'.

The authors have offered a thorough (daunting?) response to the reviews. I am more satisfied with some responses than others. But I do not think that this should stand in the way of publication in the JML special issue.

The authors acknowledge the concern about using relational notions such as `subject of the same clause' as a memory retrieval cue. They provide some text that clarifies that they need to implement a clause tracking mechanism. \textbf{I would prefer it if the text was clearer about the unfeasibility of a `subject of the same clause' feature.} But this does not impact their findings.''
\end{quote}

\subsubsection*{Response to R3.1}
\textcolor{red}{I'm not sure whether this comment requires action.}

\subsection*{Comment R3.2}
\begin{quote}
``Another concern that I raised involves the scalability of the notion that semantic interference effects reflect the use of semantic retrieval cues such as [+animate]. At issue is how this extends to finer-grained properties that are routinely responsible for plausibility violations. In my previous review I used the example of ``the teacher drove" vs. ``the little boy drove". It is, of course, clear that teachers are more plausible drivers than young children. This is part of our world knowledge. But it is less plausible that every time the noun `teacher' is encountered the property [+can drive] is activated. \textbf{The authors' response seems to be that animacy was sufficient for the materials in the current study, and that maybe other finer-grained features ``become activated only when relevant". To my mind, this point removes much of the value of the cue-based retrieval theory.} Nevertheless, this concern is about the generalizability of the authors' claims, and not about the soundness of their results.''
\end{quote}
\subsubsection*{Response to R3.2}
\textcolor{red}{I'm not sure whether this comment requires action.}

\subsection*{Comment R3.3}
\begin{quote}
``Finally, and in a similar vein, I raised questions about whether there might be other reasons for the selective interference effects found in this study. Maybe a different mechanism could capture the special status of the subject-of-the-same-clause, or maybe some specific property of the current experimental materials could be responsible for the lack of interference from other subjects. The authors seem skeptical of the first of these, and they seem more sympathetic to the second. \textbf{This is all reasonable enough, and their text conveys some caution. The abstract, on the other hand, is rather more confident.} I am sympathetic to the authors' preferred conclusion. But I am less confident than they are about how well justified it is based on the current findings. This has nothing to do with the quantitative wizardry that the authors display in the paper. It's all about the issue of generalizing from a single sentence configuration (that participants saw in 60\% of trials in the study).''
\end{quote}

\subsubsection*{Response to R3.3}
\textcolor{red}{Revise abstract to be more cautious.}

%\begin{quote}
%    \Paste{}
%\end{quote}


\end{document}  